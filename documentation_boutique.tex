\documentclass[a4paper,12pt]{article}
\usepackage[utf8]{inputenc}
\usepackage[T1]{fontenc}
\usepackage[french]{babel}
\usepackage{geometry}
\usepackage{titlesec}
\usepackage{enumitem}
\usepackage{amsmath}
\usepackage{fancyhdr}

% Configuration de la mise en page
\geometry{hmargin=2.5cm,vmargin=2.5cm}
\pagestyle{fancy}
\fancyhf{}
\rhead{Mahu - Guide Gestionnaire}
\lhead{\today}
\cfoot{\thepage}

\title{\textbf{Manuel de Procédure : Gestion des Boutiques et Produits}}
\author{Direction Technique Mahu}
\date{\today}

\begin{document}

\maketitle
\thispagestyle{empty}

\begin{abstract}
Ce document technique définit les standards professionnels pour l'ajout de boutiques, la gestion des produits avec variantes, ainsi que la structure tarifaire (commissions et frais fixes) appliquée sur la plateforme.
\end{abstract}

\tableofcontents
\newpage

\section{Administration des Boutiques (WordPress \& Dokan)}

Cette section détaille la procédure technique pour ajouter et configurer un gestionnaire de boutique via l'interface Dokan sur WordPress.

\subsection{Procédure d'Ajout d'un Vendeur}
\begin{enumerate}
    \item \textbf{Accès au Back-Office :} Connectez-vous au tableau de bord administrateur WordPress.
    \item \textbf{Création du Compte :}
    \begin{itemize}
        \item Naviguez vers l'onglet \textbf{Comptes} $\rightarrow$ \textbf{Ajouter}.
        \item Renseignez les informations : Identifiant, Email professionnel, Prénom, Nom.
        \item \textbf{Rôle Clé :} Sélectionnez \texttt{Vendeur} (Vendor) ou \texttt{Gestionnaire de boutique} dans la liste déroulante.
    \end{itemize}
    \item \textbf{Configuration Dokan :}
    \begin{itemize}
        \item Allez dans l'onglet \textbf{Dokan} $\rightarrow$ \textbf{Vendeurs}.
        \item Éditez le profil du nouveau vendeur pour activer la vente (\textit{Enable Selling}).
        \item Vérifiez les informations de la boutique (Adresse, Bannière, Logo).
    \end{itemize}
\end{enumerate}

\subsection{Validation et Conformité}
Pour garantir une image professionnelle ("Correctes et Vérifiées") :
\begin{itemize}
    \item \textbf{Vérification Légale :} Contrôle des documents d'entreprise.
    \item \textbf{Intégration API :} Configuration des clés pour l'automatisation logistique.
\end{itemize}

\section{Gestion du Catalogue Produits}

\subsection{Ajout d'un Produit}
Le gestionnaire de boutique doit fournir des informations précises pour chaque référence :
\begin{itemize}
    \item \textbf{Désignation et Description :} Textes clairs et professionnels.
    \item \textbf{Visuels :} Images haute résolution sur fond neutre.
    \item \textbf{Variantes :} Gestion des déclinaisons (Tailles, Couleurs, Matériaux) via des SKU distincts pour un suivi de stock précis.
\end{itemize}

\section{Politique Tarifaire et Commissions}

Une structure de prix automatisée est appliquée pour couvrir les frais de fonctionnement, les commissions et les coûts de conversion.

\subsection{Formule de Calcul du Prix de Vente}
Le prix final affiché au client ($P_{final}$) est calculé automatiquement à partir du prix vendeur ($P_{vendeur}$) selon la formule suivante :

\begin{equation}
    P_{final} = P_{vendeur} + \text{Commission}_{\%} + \text{Frais}_{fixe}
\end{equation}

\subsubsection{Détails des Composantes}
\begin{description}
    \item[Commission Proportionnelle :] Un taux de \textbf{7\%} est appliqué sur le prix du produit pour couvrir les frais de service.
    \[ \text{Commission}_{\%} = P_{vendeur} \times 0,07 \]
    
    \item[Frais de Conversion :] Un montant fixe de \textbf{3 USD} (ou équivalent local) est ajouté pour couvrir les frais de conversion de devise et de traitement international.
    \[ \text{Frais}_{fixe} \approx 3,00 \$ \]
\end{description}

\section{Logistique et Livraison}

La livraison est gérée de manière transparente grâce à l'intégration API configurée au niveau de la boutique.
\begin{itemize}
    \item Le calcul des frais de port est dynamique lors du paiement.
    \item L'expédition est déclenchée automatiquement après validation de la commande.
\end{itemize}

\end{document}